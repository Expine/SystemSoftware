\documentclass[11pt,a4paper]{jsarticle}
%
\usepackage{amsmath,amssymb}
\usepackage{bm}
\usepackage{graphicx}
\usepackage{ascmac}
%
\setlength{\textwidth}{\fullwidth}
\setlength{\textheight}{40\baselineskip}
\addtolength{\textheight}{\topskip}
\setlength{\voffset}{-0.2in}
\setlength{\topmargin}{0pt}
\setlength{\headheight}{0pt}
\setlength{\headsep}{0pt}
%
\title{システムソフトウェア 大課題}
\author{有松 優 \\ 学籍番号 15\_00644}
\date{\today}
%
\begin{document}
\maketitle
%
\hrulefill
\section{大課題のテーマ}
ユーザレベルスレッドライブラリの構築を行う。今回実装した機能は、以下のとおりである。
\begin{enumerate}
	\item 関数new\_threadによって、特定の関数から始まるスレッドを生成できる
	\item 関数start\_threadsによって、生成したスレッドを起動することができる
	\item スレッドの切り替えを関数yieldによって陽に行うことができる
	\item 関数th\_exitによって、スレッドを停止させることができる
	\item スレッドに指定された関数が終了した場合、スレッドが停止する
	\item wait\_thread、notify、notify\_allにより、同期機構が実装されている
\end{enumerate}


\section{ビルド方法}
使用した言語はCであり、ライブラリなどは標準ライブラリ以外用いていない。環境は、演習室のMac(macOS 10.13)とする。\\
 該当ディレクトリ内で「make」と打ち込むとビルドされる。

\section{実行方法}
以下の実行ファイルにより、それぞれ対応する処理を確認することができる。
\begin{description}
 \item[thtest1]スレッドの生成と起動、yieldによる陽な切り替え
 \item[thtest2]スレッドの停止
 \item[thtest3]wait\_thread、notify、notify\_allによる同期機構
\end{description}
 それぞれ、ビルドした後、「./thtest1」などと打ち込むと実行される。

\section{実行例}
\subsection{thtest1}
関数new\_threadによってスレッドを生成して、関数start\_threadでスレッドを起動し、関数yieldによって3つのスレッドを陽に切り替える。\\
 foo $\to$ bar $\to$ baz の順番でスレッドが切り替わる。
\begin{screen}
\begin{verbatim}
\end{verbatim}
\end{screen}

\subsection{thtest2}
関数fooを指定したスレッドは、$c$の値が$60000$を越えると関数th\_exitによりスレッドを明示的に終了させる。\\
 関数barを指定したスレッドは、一度関数printfを呼んだ後、関数barそのものが終了する。このため、スレッドが非明示的に終了する。\\
 foo $\to$ bar $\to$ baz $\to$ foo $\to$ baz $\to$ ... $\to$ foo $\to$ baz $\to$ bazの順番でスレッドが切り替わる。
\begin{screen}
\begin{verbatim}
\end{verbatim}
\end{screen}

\subsection{thtest3}
この実行例は、以下のように遷移する。
\begin{enumerate}
	\item 関数fooは、共有変数th\_tmp1(以下、単に共有変数と記述する)について、ロックする。
	\item 関数barは、共有変数を引数$c$だけ増やし、コンソールに出力。陽にスレッドを切り替える。
	\item 関数bazは、共有変数についてロックする。
	\item 関数fooはSLEEP中のため、関数barが次のスレッドになる。共有変数を引数$c$だけ増やし、コンソールに出力。その後、関数notifyを使って、関数bazを指定して起こす。
	\item 起こされた関数bazに制御が移り、共有変数を引数$c$だけ増やし、コンソールに出力。その後、関数notify\_allを使って、共有変数についてロックしているスレッドを一つ起こす。ここでは、関数fooになる。
	\item 起こされた関数fooに制御が移り、共有編すうを引数引数$c$だけ増やし、コンソールに出力。1に戻る。
\end{enumerate}

\begin{screen}
\begin{verbatim}
\end{verbatim}
\end{screen}

\section{実装上の工夫}
\subsection{スレッドの実装}
生成したスレッドを明示的に指定せずにyieldで切り替えるために、スレッドのリストを実装した。これにより、スレッドは生成した順序で順繰りに切り替えられる。これは、スレッドの数と現在処理中のスレッドを覚えておくことで実現されている。\\
\subsection{スレッド資源の開放}
スレッド資源の開放のためには、メモリアロケーションを行ったアドレスを指定する必要がある。しかし、アロケートされた開始アドレスと、スレッドに記録されたアドレスは異なっている。これは、スレッドに記録されたアドレスには、最初の関数のために存在していた引数や、スレッドの切り替え時の避難用のスタックが積まれているためである。\\
 ここでは、処理を簡便にするために、メモリをアロケートした時のアドレスを記録した。この記録したアドレスに関して関数freeで解放すればよい。
\subsection{スレッドの状態の管理}
スレッドの状態は、今回は以下の三つである。
\begin{enumerate}
	\item 実行可能状態
	\item 睡眠状態
	\item 死亡状態
\end{enumerate}
スレッドの切り替えの際、実行可能状態でなければ次のスレッドを選ぶようにすれば、実行可能状態以外のスレッドが走ることはない。\\
ここで、状態の管理は簡単のためスレッドのリストと同じインデックスに状態を格納したリストを作ることとした。関数th\_exitを読んだ場合は死亡とし、関数wait\_threadを読んだ場合は睡眠とし、関数notify、notify\_allによって再び実行可能状態とすればよい。
\subsection{同期機構の実現}
関数wait、notifyによる同期機構では、起こすスレッドが非決定的である時、飢餓が発生する可能性がある。したがって、起こすスレッドを公平に決める必要がある。ここでは、眠っているスレッドをキューで管理することで実現した。\\
 眠っているスレッドとその条件変数のペアをキューで管理する。眠るときは、キューの最後尾に挿入する。起こすときは、条件変数とスレッドの比較を行い、最初に該当したスレッドを起こせばよい。こうすることで、飢餓が発生しない同期機構が実装された。\\





%
%
\end{document}
