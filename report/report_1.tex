\documentclass[11pt,a4paper]{jsarticle}
%
\usepackage{amsmath,amssymb}
\usepackage{bm}
\usepackage{graphicx}
\usepackage{ascmac}
%
\setlength{\textwidth}{\fullwidth}
\setlength{\textheight}{40\baselineskip}
\addtolength{\textheight}{\topskip}
\setlength{\voffset}{-0.2in}
\setlength{\topmargin}{0pt}
\setlength{\headheight}{0pt}
\setlength{\headsep}{0pt}
%
\title{システムソフトウェア 大課題}
\author{有松 優 \\ 学籍番号 15\_00644}
\date{\today}
%
\begin{document}
\maketitle
%
\hrulefill
\section{大課題のテーマ}
ユーザレベルスレッドライブラリの構築を行う。今回実装した機能は、以下のとおりである。\\
\begin{enumerate}
	\item 関数new_threadによって、特定の関数から始まるスレッドを生成できる
	\item 関数start_threadsによって、生成したスレッドを起動することができる
	\item スレッドの切り替えを関数yieldによって陽に行うことができる
	\item 関数th_exitによって、スレッドを停止させることができる
\end{enumerate}


\section{ビルド方法}
使用した言語はCであり、ライブラリなどは標準ライブラリ以外用いていない。環境は、演習室のMac(macOS 10.13)とする。\\
該当ディレクトリ内で「make」と打ち込むとビルドされる。

\section{実行方法}
以下のファイルが、それぞれ対応する処理を確認することができる。
\begin{description}
 \item[thtest1]スレッドの生成と起動、yieldによる陽な切り替え
 \item[thtest2]スレッドの停止
\end{description}

\section{実行例}
\subsection{thtest1}
\begin{screen}
\begin{verbatim}
\end{verbatim}
\end{screen}

\subsection{thtest2}
\begin{screen}
\begin{verbatim}
\end{verbatim}
\end{screen}

\section{実装上の工夫}

%
%
\end{document}
